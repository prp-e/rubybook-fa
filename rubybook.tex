\documentclass[11pt]{article}
\usepackage{xepersian}
\settextfont{Nazli Bold}
\setlatintextfont{Ubuntu}
\setdigitfont{Ubuntu}
\title{آموزش زبان برنامه نویسی Ruby}
\author{محمدرضا حقیری}
\begin{document}
\maketitle
\newpage{}
\tableofcontents
\newpage{}
\newpage{}
\section{مقدمه}
زبان روبی ، یک زبان اسکریپتی، شی گرا، تابعی و مدرن است. این زبان توسط یوکیهیرو ماتسوموتو
\LTRfootnote{Yukihiro Matsumoto}
ایجاد شده و هم اکنون توسعه دهندگانی از سراسر جهان، آن را توسعه میدهند. این زبان، نحو
\LTRfootnote{Syntax}
ساده ای داشته، و می توان در عرض چند روز آن را به خوبی فرا گرفت. این زبان، برای توسعه برنامه های کاربردی دسکتاپ، وب و حتی نوشتن کتابخانه و سرویس های مختلف، کاربرد دارد. شما میتوانید به سادگی و با استفاده از این کتاب، این زبان را بیاموزید. 
\subsection{این کتاب برای چه کسانی است؟}
به اختصار، باید گفت همه علاقمندان به یادگیری این زبان ساده و جذاب. جواب طولانی تر، اینست که این کتاب، برای کسانی است که واقعا میخواهند به سراغ روبی بروند، و چون روبی مرجع فارسی مطمئنی ندارد، پشیمان می شوند. پس اگر از هر دو دسته هستید، توصیه میکنم این کتاب را از دست ندهید. 

\subsection{آیا میتوانیم این کتاب را به اشتراک بگذاریم؟}
این کتاب، رایگان عرضه شده است، همچنین شما نیز میتوانید «بدون قصد تجاری»، آن را با دوستان و خانواده خود به اشتراک بگذارید، یا برای دانلود در وبلاگ و وبسایت شخصی خود قرار دهید. 
\subsection{چگونه این کتاب را بخوانیم؟}
برای خواندن این کتاب، دو راه دارید، یا صرفا مطالعه روزنامه وار انجام دهید تا به آشنایی اجمالی با نحو زبان روبی برسید، یا اینکه تک تک نرم افزارها و مراحل گفته شده در کتاب را مرحله به مرحله انجام دهید، تا کاملا به روبی مسلط شوید. 
\newpage{}
\section{پایه ها و مقدمات یادگیری روبی}
در این فصل، چه فراخواهید گرفت؟
\begin{itemize}
\item نصب روبی در سیستم عامل
\item آشنایی با محیط اینتراکتیو روبی
\item ایجاد چند مثال ساده با روبی
\item اجرای اسکریپت نوشته شده در روبی
\end{itemize}
این موارد، اولین قدم های شما در یادگیری زبان روبی هستند. در فصول بعدی، با ساختارها و کدهای بیشتری آشنا خواهید شد، و طبیعتا آن زمان است که میتوانید از روبی، استفاده کاربردی کنید. 
\subsection{نصب روبی}
برای نصب روبی، شما ابتدا باید نام و نسخه سیستم عامل خود را بدانید. در این کتاب، فرض بر آنست که شما از گنو/لینوکس و توزیع اوبونتو استفاده میکنید.
\footnote{تفاوتی میانی استفاده از روبی در ویندوز و لینوکس نیست، طبیعتا میتوانید همین دستوراتی که در این کتابچه موجودند را به کار ببرید}
\subsubsection{وبسایت روبی}
شما میتوانید با مراجعه به وبسایت روبی 
\LTRfootnote{http://ruby-lang.org}
نسخه مورد نظر روبی را دانلود نموده، سپس آن را کامپایل کنید.
\footnote{وقتی روبی را از سورس کامپایل میکنید، باید پیشاپیش هر چه پیش نیاز آن است را دستی نصب کنید. چنانچه دانش کافی در زمینه کامپایل ندارید، توصیه میکنم از مخازن استفاده کنید.}
\subsubsection{نصب از مخازن}
برای استفاده از مخازن اوبونتو، کافیست تا یک پنجره ترمینال باز کنید، و سپس دستورات زیر را اجرا کنید :
\begin{latin}
\begin{verbatim}
sudo apt-get install ruby
\end{verbatim}
\end{latin}
همچنین اگر مفسر
\LTRfootnote{Interpreter}
دیگری در نظر دارید، میتوانید در صورت موجود بودن در مخازن، آن را نصب کنید. 
\subsection{استفاده از محیط تعاملی}
محیط تعاملی
\LTRfootnote{Interactive}
روبی، به شما این امکان را میدهد تا بدون نیاز به نوشتن کد در یک ویرایشگر متن، آن را مستقیما در ترمینال تایپ نموده، سپس نتیجه برنامه را در ترمینال ببینید. برای دسترسی به محیط تعاملی، یک پنجره ترمینال باز کرده و مانند دستور زیر عمل کنید :
\begin{latin}
\begin{verbatim}
prp-e@prp-e ~ $ irb
\end{verbatim}
\end{latin}
پس از اجرای دستور فوق، باید یک پرامپت
\LTRfootnote{Prompt}
به این شکل مشاهده کنید :
\begin{latin}
\begin{verbatim}
irb(main):001:0> 
\end{verbatim}
\end{latin}
درون این محیط، میتوانید دستورات مورد نظر را اجرا و تست کنید. 
\subsection{چند دستور در محیط تعاملی}
مثال معروفی به نام «سلام دنیا» در تمامی زبان های برنامه نویسی وجود دارد. بیایید این مثال، اولین مثال ما نیز در زبان روبی باشد. 
اکنون، پنجره ترمینال را باز کرده، و همانند شکل زیر دستور را درون محیط تعاملی روبی، وارد کنید :
\begin{latin}
\begin{verbatim}
print "Hello, World\n"
\end{verbatim}
\end{latin}
در صورتی که کد را درست وارد کرده باشید، باید نتیجه به این شکل گرفته باشید :
\begin{latin}
\begin{verbatim}
irb(main):001:0> print "Hello, World\n"
Hello, World
=> nil
irb(main):002:0> 
\end{verbatim}
\end{latin}
اما نکته اینجاست، در زبان روبی، اصولا نیازی به دستوری برای ایجاد خط جدید نداریم. دستور مناسبتری وجود دارد که جایگزین print است و می توان به کمک آن، به خوبی داده ها را آنگونه که میخواهیم، چاپ کنیم. برای انجام کار عملی، دستورات زیر را در محیط تعاملی اجرا کنید :
\begin{latin}
\begin{verbatim}
puts "Hello, World"
\end{verbatim}
\end{latin}
اکنون باید نتیجه ای مشابه این دریافت کنید :
\begin{latin}
\begin{verbatim}
irb(main):002:0> puts "Hello, World"
Hello, World
=> nil
irb(main):003:0> 
\end{verbatim}
\end{latin}
دستور puts جایگزین خوبی برای print است. علی الخصوص وقتی نتیجه مورد نظر، تنها یک خط باشد. اگرچه در آینده، از هر دو دستور استفاده خواهیم کرد. 
\subsection{نوشتن اسکریپت در یک فایل}
روبی یک زبان اسکریپتی است
\footnote{زبان های اسکریپتی، خط به خط اجرا می شوند. برخلاف زبانهای کامپایل شده، که ابتدا کل برنامه به یک زبان قابل فهم برای سیستم عامل یا ماشین ترجمه می شود، سپس اجرا میگردد. }
پس میتوانیم یک اسکریپت بنویسیم و با استفاده از مفسر، آن را اجرا کنیم. یا حتی می توانیم به آن یک دسترسی اجرایی بدهیم، سپس مستقیما اجرایش کنیم. 
\subsubsection{انتخاب ویرایشگر}
شما برای آن که بتوانید کد خود را درون یک فایل روبی بنویسید، طبیعتا به یک ویرایشگر متن
\LTRfootnote{Text Editor}
نیاز دارید. ویرایشگر های متن که همراه سیستم عامل ها نصب می شوند، گزینه های خوبی هستند. اگر میخواهید بدانید من از چه استفاده میکنم، از nano به عنوان ویرایگشر تحت ترمینال، و از gedit به عنوان ویرایشگر گرافیکی استفاده میکنم.
\footnote{در نظر داشته باشید چنانچه میزکار MATE را نصب دارید، ادیتور gedit با نام pluma در دسترس است. }
\subsubsection{پسوند فایل های روبی}
سورس کدهای روبی، عموما با پسوند های \lr{.rb} و یا \lr{.ruby} ساخته می شوند. توجه کنید که کامپیوتر به پسوند ها کاری ندارد، و این پسوند ها صرفا برای این ساخته شده اند که انسان بتواند با استفاده از آنها، نوع فایل را تشخیص دهد. 
\subsubsection{نوشتن و اجرای اسکریپت از طریق مفسر}
مفسر روبی، از طریق خط فرمان با نام ruby در دسترس است، پس میتوانیم به سادگی یک قطعه کد بنویسیم، آن را روی دسکتاپ قرار داده و سپس دستورات مورد نظر را درونش وارد کنیم. اکنون، یک فایل خالی روی دسکتاپ اجرا کنید و این دستور را درونش بنویسید :
\begin{latin}
\begin{verbatim}
puts "Hello, World"
\end{verbatim}
\end{latin}
با فرض آن که این برنامه، روی دسکتاپ شما با نام HelloWorld.rb ذخیره شده است، آن را اینگونه اجرا میکنیم :
\begin{latin}
\begin{verbatim}
prp-e@prp-e ~ $ ruby ~/Desktop/HelloWorld.rb 
\end{verbatim}
\end{latin}
سپس باید روی ترمینال چنین نتیجه ای را مشاهده کنیم :
\begin{latin}
\begin{verbatim}
prp-e@prp-e ~ $ ruby ~/Desktop/HelloWorld.rb 
Hello, World
prp-e@prp-e ~ $ 
\end{verbatim}
\end{latin}
این اسکریپت، از طریق مفسر روبی، اجرا شد. در قسمت بعدی، از طریق اجازه ها
\LTRfootnote{Permissions}
اسکریپت را اجرا خواهیم نمود.
\subsubsection{اجرای اسکریپت با استفاده از اجازه ها}
این بار، از خاصیت اجازه دسترسی در یونیکس، استفاده خواهیم کرد و سپس اسکریپت قسمت قبل را مستقلا اجرا خواهیم کرد. اکنون، نام HelloWorld.rb را به HelloWorld تغییر دهید و سپس با مفسر اجرایش کنید :
\begin{latin}
\begin{verbatim}
prp-e@prp-e ~ $ ruby ~/Desktop/HelloWorld
Hello, World
prp-e@prp-e ~ $ 
\end{verbatim}
\end{latin}
نتیجه تفاوتی نکرد، نباید هم بکند. مفسر دستورات خودش را تشخیص میدهد. حالا میخواهیم مفسر را درون خود برنامه جای دهیم!

کد برنامه را به این شکل ویرایش کنید :
\begin{latin}
\begin{verbatim}
#!/usr/bin/ruby

puts "Hello, World"
\end{verbatim}
\end{latin}
اکنون، مفسر را به اسکریپت خود شناسانده ایم. اکنون وقت آن است که اجازه اجرا به سورس کد خود بدهیم. اکنون فقط کافیست تا این دستور را در ترمینال اجرا کنید :
\begin{latin}
\begin{verbatim}
prp-e@prp-e ~ $ chmod +x Desktop/HelloWorld 
\end{verbatim}
\end{latin}

سپس میتوان این کد را مستقیما اجرا کرد. برای اجرای مستقیم کد کافیست دستور زیر را اجرا کنید :
\begin{latin}
\begin{verbatim}
prp-e@prp-e ~ $ Desktop/HelloWorld 
\end{verbatim}
\end{latin}
نتیجه دریافتی روی ترمینال باید به این شکل باشد :
\begin{latin}
\begin{verbatim}
prp-e@prp-e ~ $ Desktop/HelloWorld 
Hello, World
prp-e@prp-e ~ $ 
\end{verbatim}
\end{latin}
\subsection{جمع بندی}
در اینجا، باید یک جمع بندی نهایی از مباحثی که در فصل مطرح شدند، داشته باشیم. اول اینکه، فراگیری روبی بسیار آسان است، ولی باید بدانید کجا و چگونه، چه دستوری را به کار ببرید. دوم، باید بدانید که چه موقع محیط تعاملی، و چه موقع محیط ویرایشگر را برای نوشتن برنامه انتخاب کنید. در نهایت هم همیشه حواستان به چگونگی نوشتن کد در ویرایشگر باشد، چرا که کوچک ترین اشتباهی، میتواند مانع از اجرای درست برنامه شما باشد. 
\subsection{تمرینات}
همانگونه که در فروم اوبونتو قول داده بودم
\footnote{انجمن های فارسی اوبونتو، در آدرس forum.ubuntu.ir در دسترس هستند. قول این کتاب را در تاپیکی با مضمون آموزش روبی، در آن انجمن داده بودم. }
، در پایان هر فصل، تعدادی تمرین خواهم گنجاند. با انجام این تمرینات، به سادگی میتوانید دانش و توانایی خود در زمینه روبی را بسنجید.
\begin{enumerate}
\item برنامه ای بنویسید که نام خودتان را چاپ کند
\item برنامه ای بنویسید که نام دانش آموزان یک کلاس ۱۵ نفره را چاپ کند، و آن را در قالب یک اسکریپت با اجازه اجرا، اجرا کنید
\end{enumerate}
\newpage{}




\end{document}
